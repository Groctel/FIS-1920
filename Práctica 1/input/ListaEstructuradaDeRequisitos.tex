\section{Lista estructurada de requisitos}

\subsection{Requisitos de información}

\begin{enumerate}[label=RI-\arabic*]
	\item\textbf{Clientes:} Información del conjunto de todos los usuarios del sistema.
	\begin{itemize}
		\item DNI
		\item Datos personales
		\item Datos de pago
		\item Restricciones de área (zona, localidad, provincia)
		\item Arrendador o arrendatario
	\end{itemize}
	\item\textbf{Propiedades:} Información del conjunto de todas las propiedades en alquiler.
	\begin{itemize}
		\item Identificador
		\item Localización
		\item Superficie
		\item Edificabilidad
		\item Habitabilidad
		\item Arrendatario
		\item Arrendador
	\end{itemize}
\end{enumerate}

\subsection{Requisitos funcionales}

\subsection{Requisitos no funcionales}

\subsubsection{Usabilidad}

\begin{enumerate}
	\item La aplicación se basará en los estándares de diseño de Google para ofrecer una navegación intuitiva a usuarios primerizos.
	\item Las aplicaciones incorporarán opciones de ayuda al navegamiento para discapacitados.
	\item Se proporcionará un manual de uso para los agentes inmobiliarios que manejen el sistema.
	\item La aplicación incorporará un tutorial interactivo tanto para arrendadores como arrendatarios para guiarlos a través de los procesos básicos.
\end{enumerate}

\subsubsection{Fiabilidad}

\begin{enumerate}
	\item Se realizarán copias de seguridad periódicas de las bases de datos.
	\item Los contratos se almacenarán en sistemas que prevengan su pérdida.
\end{enumerate}

\subsubsection{Rendimiento}

\begin{enumerate}
	\item Las búsquedas de propiedades se realizarán en tiempo constante.
	\item El sistema deberá actualizar en tiempo real la disponibilidad de las propiedades.
\end{enumerate}

\subsubsection{Soporte}

\begin{enumerate}
	\item El sistema incorporará una aplicación CLI para servidores sin cabeza.
\end{enumerate}

\subsubsection{Implementación}

\begin{enumerate}
	\item Se debe usar el lenguaje C++ con bibliotecas de compatibilidad con Android e iOS\@.
	\item Las bases de datos se implementarán en MySQL\@.
\end{enumerate}

\subsubsection{Interfaz}

\begin{enumerate}
	\item El sistema enviará periódicamente sus propiedades disponibles a páginas de alquiler generalistas para que se muestren en éstas.
\end{enumerate}

\subsubsection{Legales}

\begin{enumerate}
	\item La información personal de los clientes se mantendrá privada en todo momento.
	\item Los contratos serán privados y confidenciales para usuario no implicados en ellos.
\end{enumerate}

\section{Glosario}

\begin{itemize}
	\item\textbf{Arrendador:} Persona que pone en alquiler una propiedad de cualquier tipo.
	\item\textbf{Arrendatario:} Persona que alquila una propiedad de cualquier tipo.
	\item\textbf{Propiedad:} Bien inmueble que se arrenda o es arrendado.
	\begin{itemize}
		\item\textbf{Vivienda:} Propiedad habilitada para residencia (posee cédula de habitabilidad).
		\item\textbf{Local comercial:} Propiedad habilitada para actividades comerciales (no posee cédula de habitabilidad).
	\end{itemize}
	\item\textbf{Casero:} Arrendador de una vivienda.
	\item\textbf{Inquilino:} Arrendatario de una vivienda.
	\item\textbf{Fianza:} Importe extra al primer pago utilizado como aval para el inquilino y devuelto al final del contrato tras la resta de posibles penalizaciones.
	\item\textbf{Intercambio de viviendas/Intercambio:} Proceso mediante el cual dos propietarios arrendadores arrendan mutuamente sus respectivas viviendas durante un periodo no superior a tres meses.
\end{itemize}
