\section{Lista estructurada de requisitos}

\subsection{Requisitos de información}

\begin{enumerate}[label=\textbf{RI-\arabic*}]
	\item\textbf{Clientes:} Información del conjunto de todos los usuarios del sistema.
	\begin{itemize}
		\item DNI
		\item Datos personales
		\item Datos de pago
		\item Restricciones de área (zona, localidad, provincia)
		\item Arrendador o arrendatario
	\end{itemize}
	\item\textbf{Propiedades:} Información del conjunto de todas las propiedades en alquiler.
	\begin{itemize}
		\item Identificador
		\item Localización
		\item Superficie
		\item Edificabilidad
		\item Habitabilidad
		\item Arrendatario
		\item Arrendador
	\end{itemize}
\end{enumerate}

\subsection{Requisitos funcionales}

\begin{enumerate}[label=\textbf{RF-\arabic*}]
	\item\textbf{Gestión de arrendadores:} Se permitirá dar de alta o de baja a un arrendador, así como consultar y/o modificar las propiedades que tiene, y su estado en el sistema.
	\begin{enumerate}[label=\textbf{RF-1.\arabic*}]
		\item El sistema llevará un control de los arrendadores y los inmuebles asociados a cada uno.
		\item Necesitamos una lista de los arrendadores con su información de contacto, DNI e IBAN de la cuenta bancaria en la que ingresar los alquileres, además de si pertenecen o no al sistema de intercambios.
		\begin{enumerate}[label=\textbf{RF1.2.\arabic*}]
			\item Cada uno tendrá además una lista de deseos de viviendas en el programa de intercambio, en caso de pertenecer a éste.
		\end{enumerate}
		\item Alta. Se registra un nuevo arrendador, con sus datos correspondientes
		\item Baja. Se da de baja un arrendador, se borran todos sus datos excepto los datos de contacto e IBAN\@.
		\item Baja total: Se da de baja un arrendador, se borran todos sus datos.
		\item Consultar datos de arrendador: Se muestran los datos de un arrendador existente.
		\item Modificar datos de arrendador: Se modifican los datos almacenados del arrendador correspondiente.
	\end{enumerate}
	\item\textbf{Gestión de Propiedades:} El sistema deberá llevar una gestión tanto de las propiedades disponibles en el sistema como de su estado de alquiler.
	\begin{enumerate}[label=\textbf{RF-2.\arabic*}]
		\item El sistema permitirá llevar un control de las viviendas de los arrendadores y de los arrendatarios asociados a cada una.
		\begin{enumerate}[label=\textbf{RF-2.1.\arabic*}]
			\item Necesitamos una lista de viviendas. Cada inmueble alquilado tendrá además un arrendatario asignado de la lista de clientes de la agencia.
			\item Añadir viviendas a la lista
			\item Retirar viviendas de la lista
		\end{enumerate}
		\item Ver el estado en el que se encuentra cada propiedad (Intercambiada, Alquilada, Disponible, Alquilada con impago (número de meses impagados))
		\begin{enumerate}[label=\textbf{RF-2.2.\arabic*}]
			\item Consulta de estado
			\item Modificar estado
		\end{enumerate}
		\item Ver y modificar el alquiler asociado al inmueble.
		\item Cuando una vivienda arrendada supere el límite de meses impagados, se le enviará una alerta al arrendador y se le ofrecerá la opción de efectuar un desahucio.
		\item Ver y modificar el anuncio asociado al inmueble.
	\end{enumerate}
	\item\textbf{Gestión de clientes:} El sistema llevará un control de los clientes y de los inmuebles alquilados a cada uno, en caso de estar en situación de arrendamiento. En caso de no estarlo, tendrán una lista de inmuebles deseados.
	\begin{enumerate}[label=\textbf{RF-3.\arabic*}]
		\item Se requiere una lista de los arrendatarios con su DNI, información de contacto, estado (en alquiler o buscándolo), vivienda asociada en caso de estar en alquiler, y tipo de contrato de alquiler.
		\begin{enumerate}[label=\textbf{RF-3.1.\arabic*}]
			\item Dar de alta a un nuevo cliente.
			\item Dar de baja a un nuevo cliente
			\item Consultar datos del cliente, disponibles en la lista.
		\end{enumerate}
	\end{enumerate}
	\item\textbf{Gestión de Agentes Inmobiliarios:} El sistema debe gestionar los agentes que trabajan para la empresa, vacantes y CVs de candidatos.
	\begin{enumerate}[label=\textbf{RF-4.\arabic*}]
		\item Dar de alta a un agente
		\item Dar de baja a un agente, por la razón que sea
		\item Los agentes necesitarán acceso a la lista de clientes que se encuentren buscando inmuebles, y a su lista de viviendas deseadas.
		\item Consultar sueldo de agente
		\item Modificar sueldo de agente
		\item Existirá una lista de CVs, a la que se dará uso en caso de haber puestos de trabajo disponibles
		\item Lista de clientes llevados por cada agente
		\item Número de ventas existosas en el presente meses
	\end{enumerate}
	\item\textbf{Publicidad:} El sistema tendrá acceso a la lista de deseos de los clientes/propietarios, y periódicamente les ofrecerá ponerse en contacto con un agente disponible para presentarle el inmueble.
		\begin{enumerate}[label=\textbf{RF-5.\arabic*}]
		\item Se le dará la opción al cliente de activar y desactivar dichas alertas, tanto global como individualmente.
			\begin{enumerate}[label=\textbf{RF-5.1.\arabic*}]
				\item Si la vivienda deseada es de intercambio, se le ofrecerá entrar en contacto con el propietario en lugar de con un agente.
			\end{enumerate}
		\item El sistema publicitará a los clientes viviendas relacionadas con los inmuebles de su lista de deseos.
		\item Si el cliente es nuevo o no se poseen datos porque el cliente así lo ha requerido, se publicitarán inmuebles acordes a el tipo de vivienda más popular en ese momento.
		\end{enumerate}
\end{enumerate}

\subsection{Requisitos no funcionales}

\subsubsection{Usabilidad}

\begin{enumerate}
	\item La aplicación se basará en los estándares de diseño de Google para ofrecer una navegación intuitiva a usuarios primerizos.
	\item Las aplicaciones incorporarán opciones de ayuda al navegamiento para discapacitados.
	\item Se proporcionará un manual de uso para los agentes inmobiliarios que manejen el sistema.
	\item La aplicación incorporará un tutorial interactivo tanto para arrendadores como arrendatarios para guiarlos a través de los procesos básicos.
\end{enumerate}

\subsubsection{Fiabilidad}

\begin{enumerate}
	\item Se realizarán copias de seguridad periódicas de las bases de datos.
	\item Los contratos se almacenarán en sistemas que prevengan su pérdida.
\end{enumerate}

\subsubsection{Rendimiento}

\begin{enumerate}
	\item Las búsquedas de propiedades se realizarán en tiempo constante.
	\item El sistema deberá actualizar en tiempo real la disponibilidad de las propiedades.
\end{enumerate}

\subsubsection{Soporte}

\begin{enumerate}
	\item El sistema incorporará una aplicación CLI para servidores sin cabeza.
\end{enumerate}

\subsubsection{Implementación}

\begin{enumerate}
	\item Se debe usar el lenguaje C++ con bibliotecas de compatibilidad con Android e iOS\@.
	\item Las bases de datos se implementarán en MySQL\@.
\end{enumerate}

\subsubsection{Interfaz}

\begin{enumerate}
	\item El sistema enviará periódicamente sus propiedades disponibles a páginas de alquiler generalistas para que se muestren en éstas.
\end{enumerate}

\subsubsection{Legales}

\begin{enumerate}
	\item La información personal de los clientes se mantendrá privada en todo momento.
	\item Los contratos serán privados y confidenciales para usuarios no implicados en ellos.
\end{enumerate}
